%%%%%%%%%%%%%%%%%%%%%%%%%%%%%%%%%%%%%%%%%
% Twenty Seconds Resume/CV
% LaTeX Template
% Version 1.0 (14/7/16)
%
% This template has been downloaded from:
% http://www.LaTeXTemplates.com
%
% Original author:
% Carmine Spagnuolo (cspagnuolo@unisa.it) with major modifications by 
% Vel (vel@LaTeXTemplates.com) and Harsh Gadgil
%
% License:
% The MIT License (see included LICENSE file)
%
%%%%%%%%%%%%%%%%%%%%%%%%%%%%%%%%%%%%%%%%%

%----------------------------------------------------------------------------------------
%	PACKAGES AND OTHER DOCUMENT CONFIGURATIONS
%----------------------------------------------------------------------------------------

\documentclass[letterpaper]{twentysecondcv} % a4paper for A4
\usepackage{datapie}
% Command for printing skill progress bars
\newcommand\skills{ 
~	
	\smartdiagram[bubble diagram]{
        \textbf{Quant}\\\textbf{Data Scientist},
		\textbf{Angular}\\\textbf{d3},
		\textbf{~~~Tableau~~~},
		\textbf{Pandas},
        \textbf{~~POCs~},
        \textbf{Matplotlib}\\\textbf{Seaborn},
   		\textbf{Hadoop},
        \textbf{Full}\\\textbf{~~~~~Stack~~~~~}
    }
}

\interests{{Machine Learning/0.5},{Financial Engineering/2},{Quantitative Modelling/5.5},{Data Visualization/3},{Big Data/2.5}}

% Adjusts the size of the wheel:
\def\innerradius{1.0cm}
\def\outerradius{2.2cm}

% The main macro
\newcommand{\wheelchart}[1]{
    % Calculate total
    \pgfmathsetmacro{\totalnum}{0}
    \foreach \value/\colour/\name in {#1} {
        \pgfmathparse{\value+\totalnum}
        \global\let\totalnum=\pgfmathresult
    }

    \begin{tikzpicture}

      % Calculate the thickness and the middle line of the wheel
      \pgfmathsetmacro{\wheelwidth}{\outerradius-\innerradius}
      \pgfmathsetmacro{\midradius}{(\outerradius+\innerradius)/2}

      % Rotate so we start from the top
      \begin{scope}[rotate=90]

      % Loop through each value set. \cumnum keeps track of where we are in the wheel
      \pgfmathsetmacro{\cumnum}{0}
      \foreach \value/\colour/\name in {#1} {
            \pgfmathsetmacro{\newcumnum}{\cumnum + \value/\totalnum*360}

            % Calculate the percent value
            \pgfmathsetmacro{\percentage}{\value/\totalnum*100}
            % Calculate the mid angle of the colour segments to place the labels
            \pgfmathsetmacro{\midangle}{-(\cumnum+\newcumnum)/2}

            % This is necessary for the labels to align nicely
            \pgfmathparse{
               (-\midangle<180?"west":"east")
            } \edef\textanchor{\pgfmathresult}
            \pgfmathsetmacro\labelshiftdir{1-2*(-\midangle>180)}

            % Draw the color segments. Somehow, the \midrow units got lost, so we add 'pt' at the end. Not nice...
            \fill[\colour] (-\cumnum:\outerradius) arc (-\cumnum:-(\newcumnum):\outerradius) --
            (-\newcumnum:\innerradius) arc (-\newcumnum:-(\cumnum):\innerradius) -- cycle;

            % Draw the data labels
            \draw  [*-,thin] node [append after command={(\midangle:\midradius pt) -- (\midangle:\outerradius + 1ex) -- (\tikzlastnode)}] at (\midangle:\outerradius + 1ex) [xshift=\labelshiftdir*0.5cm,inner sep=0pt, outer sep=0pt, ,anchor=\textanchor]{\name};


            % Set the old cumulated angle to the new value
            \global\let\cumnum=\newcumnum
        }

      \end{scope}
%      \draw[gray] (0,0) circle (\outerradius) circle (\innerradius);
    \end{tikzpicture}
}

%----------------------------------------------------------------------------------------
%	 PERSONAL INFORMATION
%----------------------------------------------------------------------------------------

% If you don't need one or more of the below, just remove the content leaving the command, e.g. \cvnumberphone{}



\cvname{Lucas Durand} % Your name
\cvjobtitle{ Technology Solutions \\ Associate} % Job title/career

\cvlinkedin{https://linkedin.com/in/lucasdurand}
\cvnumberphone{+1 416 884 1123} % Phone number
%\cvsite{} % Personal website
\cvsite{github.com/lucasdurand} % Personal website
\cvmail{lucas.durand@gmail.com} % Email address

%----------------------------------------------------------------------------------------

\begin{document}

\makeprofile % Print the sidebar

%----------------------------------------------------------------------------------------
%	 EDUCATION
%----------------------------------------------------------------------------------------
\section{Education}

\begin{twenty} % Environment for a list with descriptions
	\twentyitem
    	{2014 - 2016}
        {MSc., Theoretical Physics}
        {\href{http://www.yorku.ca/}{York University, Perimeter Institute (PI)}}
        {Toronto/Waterloo, Ontario, Canada}
        {}
	\twentyitem
    	{2009 - 2014}
        {BSc., Physics \& Philosophy}
        {\href{https://www.utoronto.ca/}{Trinity College, University of Toronto}}
        {Toronto, Ontario, Canada}
        {Senior Thesis -- Graphene: Hartree-Fock Analytics and Numerics}
	%\twentyitem{<dates>}{<title>}{<organization>}{<location>}{<description>}
\end{twenty}


\section{Research}
\begin{twenty}
	\twentyitem
    	{2014 - 2016}
        {Major Research Project}
        {\href{http://www.yorku.ca/stulin/research.html}{Tulin Research Group}}
        {\emph{inSIDious Matter}}
        {
        {\begin{itemize}
        \item Phenomenological exploration of a model for an inelastic self-interacting dark matter particle
        \item Theoretical formulation of scattering quantities by tensor calculus and numerical partial-wave analysis to determine valid parameter-space for theory.
       	\item New efficient techniques for numerically determining scattering cross-sections in the classical regime are implemented.
    \end{itemize}}
        }
\end{twenty}

%----------------------------------------------------------------------------------------
%	 EXPERIENCE
%----------------------------------------------------------------------------------------

\section{Experience}

\begin{twenty} % Environment for a list with descriptions
	\twentyitem
    	{Feb 2017 - \\ Present}
        {Solutions Developer - Associate}
        {\href{http://www.td.com/}{TD Bank Financial Group}}
        {Treasury Analytics Group}
        {
        {\begin{itemize}
        \item Thorough benchmark and stability testing of quasi-Monte Carlo methods shows marked increase in convergence speed for mortgage-backed security valuation model, allowing for Key Rate Vega and Convexity hedging
        \item Strengthened understanding of Financial Engineering and numerical analysis, transferring knowledge in a very successful \emph{Lunch n' Learn}, utilising Jupyter RISE to deliver interactive code and live animations
    \end{itemize}}
        }
        
    \twentyitem
   		{Jul 2016 - \\
   		Feb 2017}
        {Business Systems Analyst - Associate}
        {\href{http://www.td.com/}{TD Bank Financial Group}}
        {Enterprise Fraud Analytics Program}
        {
        {\begin{itemize}
        \item Subject Matter Expert (SME) for Identity \& Access Management. Developed Role-Based Access Management tools to automate RBAC reporting.
        \item Interfaced between business and Testing Centre of Excellence to facilitate continuous testing on vendor platform delivery.
        \item Global Knowledge training with IBM BigInsights hadoop stack
    \end{itemize}}
        }
        
     \twentyitem
   		{Apr 2014 - \\ Sep 2015}
        {Application Analyst}
        {\href{http://nbfm.ca/}{National Bank Financial Markets}}
        {Equities Support}
        {
        \begin{itemize}
		\item Worked from Exchange Tower trading floor, directly interfacing with traders and senior management        
        \item Created ETL framework and Tableau dashboards to provide meaningful business insights from global equities trade data
        \item Designed and implemented Angular.js Big Data POC for legal compliance reporting. Consolidated data from legacy Sybase servers into local SQLite database to increase search speeds by 
        \item Explored use cases for Bokeh, d3.js for visualising real-time market performance data
	    \end{itemize}
    	}
        
	%\twentyitem{<dates>}{<title>}{<location>}{<description>}
\end{twenty}

\end{document} 
